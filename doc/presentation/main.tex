% --- LaTeX Presentation Template - S. Venkatraman ---

% --- Set document class ---

% Remove "handout" when presenting to include pauses
\documentclass[dvipsnames, handout]{beamer}
\usetheme{default}

% Make content that is hidden by pauses "transparent"
\setbeamercovered{transparent}

% --- Slide layout settings ---

% Set line spacing
\renewcommand{\baselinestretch}{1.15}

% Set left and right text margins
\setbeamersize{text margin left=12mm, text margin right=12mm}

% Add slide numbers in bottom right corner
\setbeamertemplate{footline}[frame number]

% Remove navigation symbols
\setbeamertemplate{navigation symbols}{}

% Allow local line spacing changes
\usepackage{setspace}

% Change itemized list bullets to circles
\setbeamertemplate{itemize item}{$\bullet$}
\setbeamertemplate{itemize subitem}{$\circ$}

% --- Color and font settings ---

\usepackage{xcolor}
\usepackage{multicol}

% Slide title background color
\definecolor{background}{HTML}{ede6d8}
% Slide title text color; cor da feup: Pantone 484
\definecolor{titleText}{HTML}{9a3324}

% Set colors
\setbeamercolor{frametitle}{bg=background, fg=titleText}
\setbeamercolor{subtitle}{fg=titleText}

% Set font sizes for frame title and subtitle
\setbeamerfont{frametitle}{size=\fontsize{15}{16}}
\setbeamerfont{framesubtitle}{size=\small}

% --- Math/Statistics commands ---

% Add a reference number to a single line of a multi-line equation
% Usage: "\numberthis\label{labelNameHere}" in an align or gather environment
\newcommand\numberthis{\addtocounter{equation}{1}\tag{\theequation}}

% Shortcut for bold text in math mode, e.g. $\b{X}$
\let\b\mathbf
% Shortcut for bold Greek letters, e.g. $\bg{\beta}$
\let\bg\boldsymbol
% Shortcut for calligraphic script, e.g. %\mc{M}$
\let\mc\mathcal

% \mathscr{(letter here)} is sometimes used to denote vector spaces
\usepackage[mathscr]{euscript}

% Convergence: right arrow with optional text on top
% E.g. $\converge[p]$ for converges in probability
\newcommand{\converge}[1][]{\xrightarrow{#1}}

% Weak convergence: harpoon symbol with optional text on top
% E.g. $\wconverge[n\to\infty]$
\newcommand{\wconverge}[1][]{\stackrel{#1}{\rightharpoonup}}

% Equality: equals sign with optional text on top
% E.g. $X \equals[d] Y$ for equality in distribution
\newcommand{\equals}[1][]{\stackrel{\smash{#1}}{=}}

% Normal distribution: arguments are the mean and variance
% E.g. $\normal{\mu}{\sigma}$
\newcommand{\normal}[2]{\mathcal{N}\left(#1,#2\right)}

% Uniform distribution: arguments are the left and right endpoints
% E.g. $\unif{0}{1}$
\newcommand{\unif}[2]{\text{Uniform}(#1,#2)}

% Independent and identically distributed random variables
% E.g. $ X_1,...,X_n \iid \normal{0}{1}$
\newcommand{\iid}{\stackrel{\smash{\text{iid}}}{\sim}}

% Sequences (this shortcut is mostly to reduce finger strain for small hands)
% E.g. to write $\{A_n\}_{n\geq 1}$, do $\bk{A_n}{n\geq 1}$
\newcommand{\bk}[2]{\{#1\}_{#2}}

% Math mode symbols for common sets and spaces. Example usage: $\R$
\newcommand{\R}{\mathbb{R}}	% Real numbers
\newcommand{\C}{\mathbb{C}}	% Complex numbers
\newcommand{\Q}{\mathbb{Q}}	% Rational numbers
\newcommand{\Z}{\mathbb{Z}}	% Integers
\newcommand{\N}{\mathbb{N}}	% Natural numbers
\newcommand{\F}{\mathcal{F}}	% Calligraphic F for a sigma algebra
\newcommand{\El}{\mathcal{L}}	% Calligraphic L, e.g. for L^p spaces

% Math mode symbols for probability
\newcommand{\pr}{\mathbb{P}}	% Probability measure
\newcommand{\E}{\mathbb{E}}	% Expectation, e.g. $\E(X)$
\newcommand{\var}{\text{Var}}	% Variance, e.g. $\var(X)$
\newcommand{\cov}{\text{Cov}}	% Covariance, e.g. $\cov(X,Y)$
\newcommand{\corr}{\text{Corr}}	% Correlation, e.g. $\corr(X,Y)$
\newcommand{\B}{\mathcal{B}}	% Borel sigma-algebra

% Other miscellaneous symbols
\newcommand{\tth}{\text{th}}	% Non-italicized 'th', e.g. $n^\tth$
\newcommand{\Oh}{\mathcal{O}}	% Big-O notation, e.g. $\O(n)$
\newcommand{\1}{\mathds{1}}	% Indicator function, e.g. $\1_A$

% Additional commands for math mode
\DeclareMathOperator*{\argmax}{argmax}	% Argmax, e.g. $\argmax_{x\in[0,1]} f(x)$
\DeclareMathOperator*{\argmin}{argmin}	% Argmin, e.g. $\argmin_{x\in[0,1]} f(x)$
\DeclareMathOperator*{\spann}{Span}	% Span, e.g. $\spann\{X_1,...,X_n\}$
\DeclareMathOperator*{\bias}{Bias}	% Bias, e.g. $\bias(\hat\theta)$
\DeclareMathOperator*{\ran}{ran}		% Range of an operator, e.g. $\ran(T) 
\DeclareMathOperator*{\dv}{d\!}		% Non-italicized 'with respect to', e.g. $\int f(x) \dv x$
\DeclareMathOperator*{\diag}{diag}	% Diagonal of a matrix, e.g. $\diag(M)$
\DeclareMathOperator*{\trace}{trace}	% Trace of a matrix, e.g. $\trace(M)$
\DeclareMathOperator*{\supp}{supp}	% Support of a function, e.g., $\supp(f)$

% --- Presentation begins here ---

\begin{document}

% --- Title slide ---

\title{\color{titleText} Flu Shot Learning}
\subtitle{\color{titleText}Predicting H1N1 and Seasonal flu vaccination}
\author{Jorge Pais, José Baptista e Pedro Duarte\vspace{-.3cm}}
\institute{MsC Electrical and Computer Engineering}
\date{}

\begin{frame}
\titlepage
\vspace{-1.2cm}
\begin{center}
    \begin{spacing}{1.2}\scriptsize 
    Final project of the Machine Learning course M.EEC006 
    \end{spacing}
\end{center}
\end{frame}

% --- Main content ---

% Example slide: use \pause to sequentially unveil content
\begin{frame}{Problem Description}
\framesubtitle{Context and Objectives}

\begin{itemize}
    \item Competition hosted by DrivenData
    \item The objective is to, based on socio-economical background, opinions, concerns, e.t.c, predict if whether a individual has taken the H1N1 or Seasonal Flu vaccines
    \item Performance measurements by using Receiver Operating Characteristic (ROC)
\end{itemize}

\end{frame}

% Example slide 2: Image
\begin{frame}{Data Set}
\begin{multicols}{2}

\begin{itemize}
    \item The dataset was the 2009 National H1N1 Flu Survey, consisting of 35 attributes with both numerical (ordinal/binary) and categorical features.
    \item Two-labels to predict: \texttt{h1n1\_vaccine} and \texttt{seasonal\_vaccine}
    \item Two sets of 26707 respondents for the training data and evaluation, including and excluding labels, respectively.
\end{itemize} 
\includegraphics[width = 5cm, height = 5cm]{aaa.png}
\end{multicols}
\end{frame}


\begin{frame}{Implementation}
    \begin{multicols}{2}
        \includegraphics[width = 4.7cm]{MethodologyStructure.png}
        Several standard classification models were utilized:
        \begin{itemize}
            \item Logistic Regression
            \item Naive Bayes Classifiers
            \item Decision Trees
            \item K-Nearest Neighbors
        \end{itemize}
        \pause
        Along with some gradient boosting algorithms:
        \begin{itemize}
            \item \textit{XGBoost}
            \item \textit{CatBoost}
        \end{itemize}
    \end{multicols}
\end{frame}


\begin{frame}{Results}
    \begin{multicols*}{2}
        For the standard models, the results on folded training data:
        \includegraphics[width=5cm]{results1.png}
        And for gradient boosting:
        \includegraphics[width=5cm]{results2.png}
        Example of ROC plot:
        \includegraphics[width=5cm]{ROC.png}
    \end{multicols*}
\end{frame}
% --- Thank you slide ---

\begin{frame}{Results}
\framesubtitle{Competition}
    
    \begin{tabular}{|l|l|}
        \hline
    \textbf{Model} & \textbf{ROC AUC score} \\ 
    \hline \hline
    Logistic Regression (C=0.1) & 0.8356 \\ \hline
    K-Nearest Neighbors (k=169) & 0.8122 \\ \hline
    CatBoost with One-Hot encoding & 0.8430 \\ \hline
    CatBoost with specified cat\_features & 0.8616 \\ \hline
    \end{tabular}
    \vspace{5pt}
    \begin{center}
        Final place on the leaderboard: 285th    
    \end{center}
    
\end{frame}

\begin{frame}
\begin{center}
{\large\color{titleText} Thank you for listening!}
\vspace{1cm}
\end{center}
\end{frame}

% --- Presentation ends here ---

\end{document}
